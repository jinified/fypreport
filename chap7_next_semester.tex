\chapter{Next Semester}

\section{Existing problems}

\begin{enumerate}
    \item Very costly to acquire large training data
    \item Manual selection of algorithms and parameters based on expert knowledge required
    \item Expensive sensors needed to achieve robust tracking. Non-visual based sensors i.e LIDAR and sonar provides competitive edge
    \item State-of-the art tracking algorithm has high detection latency. Not suitable for real-time application 
\end{enumerate}

\section{Objectives}

\begin{enumerate}
    \item Robustness
        \begin{itemize}
            \item Different water bodies i.e lake, ocean
            \item Changes in illuminations
            \item Shadow
        \end{itemize}
    \item Automation
        \begin{itemize}
            \item Automatically chooses algorithm and parameters for preprocessing offline
            \item Automatically tune camera parameter to achieve optimal image for further processing
            \item Recommends object detection algorithm given image
        \end{itemize}
    \item Flexibility
        \begin{itemize}
            \item Easily adapted to different Robosub missions
            \item Algorithm scales with more training data
        \end{itemize}
    \item Rapid Prototyping \\
        Very fast iteration with limited data collection. Able to learn on the fly or suggest what data to collect
\end{enumerate}

\section{How do we measure success ?}

\begin{enumerate}
    \item Accuracy across different challenging dataset
    \item Precision of bounding box
    \item FPS
    \item Number of training data vs accuracy 
\end{enumerate}

\section{Mental blocks}

\begin{enumerate}
    \item Ensemble approach vs automatically choosing the best algorithm
    \item Focus on preprocessing vs working on color space that is not affected by illuminations and color degradation
    \item Domain adaptation / transfer learning vs independent task learning
    \item Using multiple cues and features vs simple features that just work
    \item One general framework that works for different tasks while allowing unique strategy to be built for specific task
    \item How much invariance needed to be enforced ? Affine ?
\end{enumerate}

\section{Innovations}

\begin{enumerate}
    \item Non-parametric approach
    \item Ensemble approach i.e bagging , boosting
    \item Detection free approach when near object
    \item Transfer Learning
    \item Combining local and global features
    \item Automatic mission testing 
    \item Using visual prior to encode belief
    \item Tool for rapid annotation on the fly due to competition need
    \item Synthesize data to test robustness and increase training dataset
    
    
\end{enumerate}

\section{Methodology}

\subsection{Automatic camera tuning}

\begin{enumerate}
    \item Make use of image entropy to tune parameters
    \item Using mean intensity of brightness
\end{enumerate}

\subsection{Choosing color space}

\begin{enumerate}
    \item Illumination-invariant color space
    \item Work on chromacity of the objects 
    \item Different color space for different task
\end{enumerate}

\subsection{Image enhancement}

\begin{enumerate}
    \item Contrast stretching
    \item Gamma correction 
    \item Image fusion 
    \item Histogram Equalization 
    \item Removing flicker
\end{enumerate}

\subsection{Object Proposals}
Focus on identifying all potential objects in the scene

\begin{enumerate}
    \item EdgeBox
    \item SelectiveSearch
    \item Saliency
    \item MSER
    \item BING
    \item Background Subtraction
\end{enumerate}

\subsection{Object Detection}

\begin{enumerate}
    \item HOG
    \item Integral Channel Feature
    \item Color Histogram
    \item Correlation Filter
    \item HuMoment
    \item GIST
    \item Binary Feature Detectors
    \item Haar like feature
\end{enumerate}

\subsection{Object Tracking}

\begin{enumerate}
    \item Particle Filter
    \item Mean Shift
    \item Covariance tracking
\end{enumerate}

