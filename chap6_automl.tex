\chapter{Automatic Machine Learning}

\section{Adaptive Algorithm and Platform Selection for Visual Detection
and Tracking}

\subsection{Main Ideas}

Suggests the best algorithm-parameter pair given characteristics of the
video. Separates into two phases:

\begin{enumerate}
  \item Design Phase: Learn mapping between scenario in
    training dataset with algorithm-parameter pair
  \item Runtime Phase: Calculate similarity
    between test video and training dataset to
    choose most similar dataset in database
\end{enumerate}

\subsection{Similarity measures}

\begin{enumerate}
  \item $S(T_i,R_j) =
    \epsilon^{-d(T_i,R_j)}$
    \\
    Feature-distance
    is calculated
    by the
    geodesic flow.
    Projecting
    feature onto
    Grasmann
    manifold.
    \begin{equation}
      t_i^{T}W_{ij}r_{j}
      =
      \int_{0}^{1}
      (\theta(y)t_i)^T
      (\theta(y)r_j)
      dy
    \end{equation}
    \begin{equation}
      d(T_i,R_j)
      =
      t_i^{T}W_{ij}t_i
      +
      r_j^{T}W_{ij}r_j
      -
      2t_i^{T}W_{ij}r_j
    \end{equation}

\end{enumerate}

\section{On-line
selection
of
discriminative
features
\cite{collins2005online}}

\subsection{Feature
Selection}
Mainly
consists
of:
a)
feature
selection
criterion
and
b)
search
strategy.
Evaluate
each
feature
based
on
\textit{class
separability}
from
surrounding.

\begin{enumerate}
  \item
    Estimate
    histogram
    of
    detected
    object
    and
    surrounding
    (center-surround
    principle)
  \item
    Calculate
    log-likelihood
    ratio
    of
    these
    distributions
    \[L(i)
    =
    \log
    \frac{\max{p(i),
    0.0001}}{\max{q(i),
    0.0001}}\]
  \item
    Calculate
    variance
    ratio
    of
    log-likehood 
    \[VR(L,p,q)
    =
    \frac{var(L\setminusfrac{p+q}{2})}{var(L,p)
    +
    var(L,
    q)}\]
\end{enumerate}

\section{Automatic
Parameter
Adapation
for
MOT
\cite{Chau2013}}

\subsection{Offline
Learning}

\subsubsection{Contextual
Feature
Extraction}
\begin{enumerate}
  \item
    Density
    of
    objects
  \item
    Occlusion
    level
  \item
    Contrast
    of
    object
  \item
    Contrast
    variance
    between
    objects
  \item
    Area
    of
    objects
  \item
    Area
    Variance
    between
    objects
\end{enumerate}

\subsubsection{Context
Code
book}
Divides
training
video
into
chunks
with
similar
context
code
book
\begin{enumerate}
  \item
    Code
    book
    consists
    of
    a
    set
    of
    code
    words
  \item
    Code
    word 
    \[cw^{k}
    =
    {mean
    value,
    maximum,
    minimum,
    frequencies}\]
  \item
    Context
    distance
    is
    ratio
    between
    mean
    value
    of
    code
    word
\end{enumerate}

\subsection{Online
steps}

\subsubsection{Context
detection}

Given
an
input
video
stream,
if
context
of
video
is
in
database
do
nothing,
else
activate
parameter
tuning

\subsubsection{Parameter
tuning}

Chooses
context
that
is
closest
from
trained
database

\section{Efficient
and
Robust
Automated
Machine
Learning
(winner
of
AutoML)
\cite{feurer2015efficient}}

\subsection{Main
Ideas}
Extends
from
Auto-Weka
which
uses
tree-based
Bayesian
Optimization
\cite{thornton2013auto}
to
model
relationship
between
hyperparameters
setting
and
performance.
A
random
forest
based
SMAC
approach
is
used
to
solve
CASH
problem.

\subsection{Meta-learning}
Selects
instatiations
of
machine
learning
frameworks
as
seed
for
Bayesian
Optimization
(seeding).
Train
hyperparameters
that
perform
best
on
offline
data
using
Bayesian
Optimization.

\subsection{Automated
ensemble
construction
of
models}
Uses
ensemble
selection
\cite{caruana2004ensemble}
to
greedily
add
models
learned
from
Bayesian
Optimization
to
prevent
over-fitting 

\subsection{Practical
application}
Based
on
auto-sklearn
components
\cite{NIPS2015_5872}.
Alternatives
include:
\begin{enumerate}
  \item
    Hyperopt-sklearn
    \cite{komer2014hyperopt}
  \item
    Auto-Weka
    \cite{thornton2013auto}
\end{enumerate}

\section{A
brief
review
of
ChaLearn
AutoML
Challenge
\cite{guyon2016brief}}

\subsection{Preprocessing}
\begin{enumerate}
  \item
    Zero
    mean
    and
    unit
    variance
    (ZMUV)
  \item
    non-linear
    transformation
    (log)
  \item
    dimensionality
    reduction
    (PCA,
    ICA)
\end{enumerate}

\subsection{Model
selection}
\begin{enumerate}
  \item
    K-fold
    cross-validation
  \item
    Transferring
    knowledge
    from
    phase
    to
    phase
\end{enumerate}
